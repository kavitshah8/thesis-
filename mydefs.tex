%
%  mydefs.tex  2007-03-19  Mark Senn  http://www.ecn.purdue.edu/~mark
%
%  Command definitions that can be used in all documents that have
%      %
%  mydefs.tex  2007-03-19  Mark Senn  http://www.ecn.purdue.edu/~mark
%
%  Command definitions that can be used in all documents that have
%      %
%  mydefs.tex  2007-03-19  Mark Senn  http://www.ecn.purdue.edu/~mark
%
%  Command definitions that can be used in all documents that have
%      %
%  mydefs.tex  2007-03-19  Mark Senn  http://www.ecn.purdue.edu/~mark
%
%  Command definitions that can be used in all documents that have
%      \input{mydefs}
%

% CHANGE NEXT 3 LINES?
% Define \be and \ee to start and end the equation environment.
\newcommand{\be}{\begin{equation}}
\newcommand{\ee}{\end{equation}}

\newcommand{\tree}{$\mathcal{T}$}
\newcommand{\treeRoot}{$\mathcal{R}$}
\newcommand{\querier}{$\mathcal{Q}$}
\newcommand{\children}{$children$}
\newcommand{\parent}{$\mathcal{P}$}
\newcommand{\at}{$AggregationTree$}
\newcommand{\aggregator}{$A_{N}$}

\newcommand{\node}{$\mathcal{N}$}
\newcommand{\child}{$\mathcal{C}$}
\newcommand{\vertex}{$\mathcal{V}$}
\newcommand{\complainer}{$\mathcal{C}$}
\newcommand{\cheater}{$CHEATER$}
\newcommand{\lc}{$leftChild$}
\newcommand{\rc}{$rightChild$}

\newcommand{\cert}{$cert$}
\newcommand{\sign}{$SIGN$}
\newcommand{\msg}{$msg$}
\newcommand{\forest}{$forest$}
\newcommand{\nextTree}{$nextTree$}
\newcommand{\temp}{$temp$}
\newcommand{\height}{$height$}


% CHANGE NEXT 12 LINES?
% Define \Repeat so, for example,
%     \Repeat{whatever}{10}
% is the same as typing whatever 10 times.
\newcount{\myi}
\newcommand{\Repeat}[2]{%
    \myi=0
    \loop
        \ifnum\myi<#2
        #1
        \advance\myi by 1
    \repeat
}

% CHANGE NEXT 3 LINES?
% Make "\Sum ab" or "\Sum{a}{b}" do "\sum_{a}^{b}".
% This can only be used when in math mode.
\newcommand\Sum[2]{\sum_{#1}^{#2}}

% CHANGE NEXT 4 LINES?
% Make "\xn" do "$x_n$".
% Because this definition contains the "$" to go into math mode
% this definition must be used when not in math mode.
\newcommand{\xn}{$x_n$}

% CHANGE NEXT 5 LINES?
% Since \xn is already defined we must use \renewcommand to redefine it.
% Normally you would not have the above definition for \xn in this file
% if you were just going to override it later.
% The \ensuremath goes into math mode if not already in math mode.
\renewcommand{\xn}{\ensuremath{x_n}}


%

% CHANGE NEXT 3 LINES?
% Define \be and \ee to start and end the equation environment.
\newcommand{\be}{\begin{equation}}
\newcommand{\ee}{\end{equation}}

\newcommand{\tree}{$\mathcal{T}$}
\newcommand{\treeRoot}{$\mathcal{R}$}
\newcommand{\querier}{$\mathcal{Q}$}
\newcommand{\children}{$children$}
\newcommand{\parent}{$\mathcal{P}$}
\newcommand{\at}{$AggregationTree$}
\newcommand{\aggregator}{$A_{N}$}

\newcommand{\node}{$\mathcal{N}$}
\newcommand{\child}{$\mathcal{C}$}
\newcommand{\vertex}{$\mathcal{V}$}
\newcommand{\complainer}{$\mathcal{C}$}
\newcommand{\cheater}{$CHEATER$}
\newcommand{\lc}{$leftChild$}
\newcommand{\rc}{$rightChild$}

\newcommand{\cert}{$cert$}
\newcommand{\sign}{$SIGN$}
\newcommand{\msg}{$msg$}
\newcommand{\forest}{$forest$}
\newcommand{\nextTree}{$nextTree$}
\newcommand{\temp}{$temp$}
\newcommand{\height}{$height$}


% CHANGE NEXT 12 LINES?
% Define \Repeat so, for example,
%     \Repeat{whatever}{10}
% is the same as typing whatever 10 times.
\newcount{\myi}
\newcommand{\Repeat}[2]{%
    \myi=0
    \loop
        \ifnum\myi<#2
        #1
        \advance\myi by 1
    \repeat
}

% CHANGE NEXT 3 LINES?
% Make "\Sum ab" or "\Sum{a}{b}" do "\sum_{a}^{b}".
% This can only be used when in math mode.
\newcommand\Sum[2]{\sum_{#1}^{#2}}

% CHANGE NEXT 4 LINES?
% Make "\xn" do "$x_n$".
% Because this definition contains the "$" to go into math mode
% this definition must be used when not in math mode.
\newcommand{\xn}{$x_n$}

% CHANGE NEXT 5 LINES?
% Since \xn is already defined we must use \renewcommand to redefine it.
% Normally you would not have the above definition for \xn in this file
% if you were just going to override it later.
% The \ensuremath goes into math mode if not already in math mode.
\renewcommand{\xn}{\ensuremath{x_n}}


%

% CHANGE NEXT 3 LINES?
% Define \be and \ee to start and end the equation environment.
\newcommand{\be}{\begin{equation}}
\newcommand{\ee}{\end{equation}}

\newcommand{\tree}{$\mathcal{T}$}
\newcommand{\treeRoot}{$\mathcal{R}$}
\newcommand{\querier}{$\mathcal{Q}$}
\newcommand{\children}{$children$}
\newcommand{\parent}{$\mathcal{P}$}
\newcommand{\at}{$AggregationTree$}
\newcommand{\aggregator}{$A_{N}$}

\newcommand{\node}{$\mathcal{N}$}
\newcommand{\child}{$\mathcal{C}$}
\newcommand{\vertex}{$\mathcal{V}$}
\newcommand{\complainer}{$\mathcal{C}$}
\newcommand{\cheater}{$CHEATER$}
\newcommand{\lc}{$leftChild$}
\newcommand{\rc}{$rightChild$}

\newcommand{\cert}{$cert$}
\newcommand{\sign}{$SIGN$}
\newcommand{\msg}{$msg$}
\newcommand{\forest}{$forest$}
\newcommand{\nextTree}{$nextTree$}
\newcommand{\temp}{$temp$}
\newcommand{\height}{$height$}


% CHANGE NEXT 12 LINES?
% Define \Repeat so, for example,
%     \Repeat{whatever}{10}
% is the same as typing whatever 10 times.
\newcount{\myi}
\newcommand{\Repeat}[2]{%
    \myi=0
    \loop
        \ifnum\myi<#2
        #1
        \advance\myi by 1
    \repeat
}

% CHANGE NEXT 3 LINES?
% Make "\Sum ab" or "\Sum{a}{b}" do "\sum_{a}^{b}".
% This can only be used when in math mode.
\newcommand\Sum[2]{\sum_{#1}^{#2}}

% CHANGE NEXT 4 LINES?
% Make "\xn" do "$x_n$".
% Because this definition contains the "$" to go into math mode
% this definition must be used when not in math mode.
\newcommand{\xn}{$x_n$}

% CHANGE NEXT 5 LINES?
% Since \xn is already defined we must use \renewcommand to redefine it.
% Normally you would not have the above definition for \xn in this file
% if you were just going to override it later.
% The \ensuremath goes into math mode if not already in math mode.
\renewcommand{\xn}{\ensuremath{x_n}}


%

% CHANGE NEXT 3 LINES?
% Define \be and \ee to start and end the equation environment.
\newcommand{\be}{\begin{equation}}
\newcommand{\ee}{\end{equation}}

\newcommand{\tree}{$\mathcal{T}$}
\newcommand{\treeRoot}{$\mathcal{R}$}
\newcommand{\querier}{$\mathcal{Q}$}
\newcommand{\children}{$children$}
\newcommand{\parent}{$\mathcal{P}$}
\newcommand{\aggregator}{$A_{N}$}

\newcommand{\node}{$\mathcal{N}$}
\newcommand{\child}{$\mathcal{C}$}
\newcommand{\vertex}{$\mathcal{V}$}
% \newcommand{\complainer}{$\mathcal{C}$}
\newcommand{\cheater}{$CHEATER$}
\newcommand{\lc}{$leftChild$}
\newcommand{\rc}{$rightChild$}

\newcommand{\cert}{$cert$}
\newcommand{\sign}{$SIGN$}
\newcommand{\msg}{$msg$}
\newcommand{\forest}{$forest$}
\newcommand{\nextTree}{$nextTree$}
\newcommand{\temp}{$temp$}
\newcommand{\height}{$height$}

% commitmnet tree related
\newcommand{\at}{\textit{\textbf{AggregationTree}}} 
\newcommand{\bs}{\textit{\textbf{BaseStation}}} 
\newcommand{\q}{\textit{\textbf{Querier}}} 
\newcommand{\inforate}{\textit{Inforate}}
% Definitions
\newcommand{\payload}{payload}
\newcommand{\payloads}{payloads}
\newcommand{\informationRate}{\textit{\textbf{information rate}}}

\newcommand{\mac}{$\textit{MAC}$}
\newcommand{\ack}{$\textit{ACK}$}
\newcommand{\nack}{$\textit{NACK}$}
\newcommand{\complainer}{$\textit{complainer}$}
\newcommand{\R}{${\rm I\!R}$}
\newcommand{\E}{${\rm I\!E}$}


\newtheorem{exmp}{Example}[section]
% CHANGE NEXT 12 LINES?
% Define \Repeat so, for example,
%     \Repeat{whatever}{10}
% is the same as typing whatever 10 times.
\newcount{\myi}
\newcommand{\Repeat}[2]{%
    \myi=0
    \loop
        \ifnum\myi<#2
        #1
        \advance\myi by 1
    \repeat
}

% CHANGE NEXT 3 LINES?
% Make "\Sum ab" or "\Sum{a}{b}" do "\sum_{a}^{b}".
% This can only be used when in math mode.
\newcommand\Sum[2]{\sum_{#1}^{#2}}

% CHANGE NEXT 4 LINES?
% Make "\xn" do "$x_n$".
% Because this definition contains the "$" to go into math mode
% this definition must be used when not in math mode.
\newcommand{\xn}{$x_n$}

% CHANGE NEXT 5 LINES?
% Since \xn is already defined we must use \renewcommand to redefine it.
% Normally you would not have the above definition for \xn in this file
% if you were just going to override it later.
% The \ensuremath goes into math mode if not already in math mode.
\renewcommand{\xn}{\ensuremath{x_n}}

