%
%  revised  front.tex  2011-09-02  Mark Senn  http://engineering.purdue.edu/~mark
%  created  front.tex  2003-06-02  Mark Senn  http://engineering.purdue.edu/~mark
%
%  This is ``front matter'' for the thesis.
%
%  Regarding ``References'' below:
%      KEY    MEANING
%      PU     ``A Manual for the Preparation of Graduate Theses'',
%             The Graduate School, Purdue University, 1996.
%      TCMOS  The Chicago Manual of Style, Edition 14.
%      WNNCD  Webster's Ninth New Collegiate Dictionary.
%
%  Lines marked with "%%" may need to be changed.
%

  % Dedication page is optional.
  % A name and often a message in tribute to a person or cause.
  % References: PU 15, WNNCD 332.
\begin{dedication}
  This is the dedication.
\end{dedication}

  % Acknowledgements page is optional but most theses include
  % a brief statement of apreciation or recognition of special
  % assistance.
  % Reference: PU 16.
\begin{acknowledgments}
  This is the acknowledgments.
\end{acknowledgments}

  % The preface is optional.
  % References: PU 16, TCMOS 1.49, WNNCD 927.
\begin{preface}
  This is the preface.
\end{preface}

  % The Table of Contents is required.
  % The Table of Contents will be automatically created for you
  % using information you supply in
  %     \chapter
  %     \section
  %     \subsection
  %     \subsubsection
  % commands.
  % Reference: PU 16.
\tableofcontents

  % If your thesis has tables, a list of tables is required.
  % The List of Tables will be automatically created for you using
  % information you supply in
  %     \begin{table} ... \end{table}
  % environments.
  % Reference: PU 16.
\listoftables

  % If your thesis has figures, a list of figures is required.
  % The List of Figures will be automatically created for you using
  % information you supply in
  %     \begin{figure} ... \end{figure}
  % environments.
  % Reference: PU 16.
\listoffigures

  % List of Symbols is optional.
  % Reference: PU 17.
\begin{symbols}
  $s$& Sensor node \cr
  $N$& Query nonce\cr
  $H$& Hash function\cr
  $d$& Distance \cr
  $D$& Data-item \cr
  $X$& Random variable \cr
  $\delta$& Fanout of a sensor node \cr
  $f$& Function \cr
  $v$& Vertex \cr
  $A$& An attack \cr
  $\alpha$& Resilient factor \cr
\end{symbols}

  % List of Abbreviations is optional.
  % Reference: PU 17.
\begin{abbreviations}
  SHIA& Secure hierarchical in-network aggregation\cr
  SIA& Secure Information aggregation\cr
  ACK& Positive acknowledgment message\cr
  NACK& Negative acknowledgment message\cr
\end{abbreviations}

  % Nomenclature is optional.
  % Reference: PU 17.
\begin{nomenclature}
  Alanine& 2-Aminopropanoic acid\cr
  Valine& 2-Amino-3-methylbutanoic acid\cr
\end{nomenclature}

  % Glossary is optional
  % Reference: PU 17.
\begin{glossary}
  chick& female, usually young\cr
  dude& male, usually young\cr
\end{glossary}

  % Abstract is required.
  % Note that the information for the first paragraph of the output
  % doesn't need to be input here...it is put in automatically from
  % information you supplied earlier using \title, \author, \degree,
  % and \majorprof.
  % Reference: PU 17.
\begin{abstract}
  This is the abstract.
\end{abstract}
