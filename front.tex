
%
%  revised  front.tex  2011-09-02  Mark Senn  http://engineering.purdue.edu/~mark
%  created  front.tex  2003-06-02  Mark Senn  http://engineering.purdue.edu/~mark
%
%  This is ``front matter'' for the thesis.
%
%  Regarding ``References'' below:
%      KEY    MEANING
%      PU     ``A Manual for the Preparation of Graduate Theses'',
%             The Graduate School, Purdue University, 1996.
%      TCMOS  The Chicago Manual of Style, Edition 14.
%      WNNCD  Webster's Ninth New Collegiate Dictionary.
%
%  Lines marked with "%%" may need to be changed.
%

  % Dedication page is optional.
  % A name and often a message in tribute to a person or cause.
  % References: PU 15, WNNCD 332.
% \begin{dedication}
%   This is the dedication.
% \end{dedication}

  % Acknowledgements page is optional but most theses include
  % a brief statement of apreciation or recognition of special
  % assistance.
  % Reference: PU 16.
\begin{acknowledgments}
  I would like to express my special appreciation and thanks to my advisor Brian King who has been my teacher, guide and mentor during the entire thesis and Master's program.
  I am grateful to him for being a great role model for researcher and a mentor in my life.
  I would like to thank him for encouraging my research and allowing me to grow as a researcher.
  % Your advice on both research as well as on my career have been priceless

  I would also like to thank the other members of my thesis committee: Paul Salama, Mohamed El-Sharkawy and Sangkook Lee for letting my defense be an enjoyable moment, and their brilliant comments and suggestions after the presentation. 
  % I would like to thank them for their constructive feedback during the collaboration. 

  I would also like to thank the entire academic staff of Electrical and Computer Engineering Department at Indiana University Purdue University - Indianapolis for being generous and helpful all the time. 

  Finally, I would like to thank my family for being patient and loving all the time.
  My family's financial and emotional support for me has been incredible.
  Their prayers for me was what sustained me thus far.    

\end{acknowledgments}

  % The preface is optional.
  % References: PU 16, TCMOS 1.49, WNNCD 927.
% \begin{preface}
%   This is the preface.
% \end{preface}

  % The Table of Contents is required.
  % The Table of Contents will be automatically created for you
  % using information you supply in
  %     \chapter
  %     \section
  %     \subsection
  %     \subsubsection
  % commands.
  % Reference: PU 16.
\tableofcontents

  % If your thesis has tables, a list of tables is required.
  % The List of Tables will be automatically created for you using
  % information you supply in
  %     \begin{table} ... \end{table}
  % environments.
  % Reference: PU 16.
\listoftables

  % If your thesis has figures, a list of figures is required.
  % The List of Figures will be automatically created for you using
  % information you supply in
  %     \begin{figure} ... \end{figure}
  % environments.
  % Reference: PU 16.
\listoffigures

  % List of Symbols is optional.
  % Reference: PU 17.
\begin{symbols}
  % $s$& Sensor node \cr
  $N$& Query nonce\cr
  $H$& Hash function\cr
  % $d$& Distance \cr
  % $D$& Data-item \cr
  % $X$& Random variable \cr
  % $\delta$& Fanout of a sensor node \cr
  % $f$& Function \cr
  % $v$& Vertex \cr
  % $A$& An attack \cr
  % $\alpha$& Resilient factor \cr
\end{symbols}

  % List of Abbreviations is optional.
  % Reference: PU 17.
\begin{abbreviations}
  ACK& Positive acknowledgment message\cr
  BER& Bit error rate\cr
  FSwRD& Forwarding signatures with resigning the data-items\cr
  FSwoRD& Forwarding signatures without resigning the data-items\cr
  MAC& Message authentication code\cr
  NACK& Negative acknowledgment message\cr
  SHA& Secure hash algorithm\cr
  SIA& Secure information aggregation\cr
  SHIA& Secure hierarchical in-network aggregation\cr
\end{abbreviations}

  % Nomenclature is optional.
  % Reference: PU 17.
% \begin{nomenclature}
%   Alanine& 2-Aminopropanoic acid\cr
%   Valine& 2-Amino-3-methylbutanoic acid\cr
% \end{nomenclature}

  % Glossary is optional
  % Reference: PU 17.
% \begin{glossary}
%   chick& female, usually young\cr
%   dude& male, usually young\cr
% \end{glossary}

  % Abstract is required.
  % Note that the information for the first paragraph of the output
  % doesn't need to be input here...it is put in automatically from
  % information you supplied earlier using \title, \author, \degree,
  % and \majorprof.
  % Reference: PU 17.
\begin{abstract}
  % We show that 

  % We show that two distinguishing properties of sensor networks, i.e., the presence of a trusted base station, and the pre-knowledge of the fixed network topology, can yield security protocols that are both communication-efficient and highly general. 
  
  We propose a secure in-network data aggregation protocol with internal verification, to gain increase in the lifespan of the network by preserving bandwidth.
  For doing secure internalw distributed operations, we show an algorithm for securely computing the sum of sensor readings in the network.
  Our algorithm can be generalized to any random tree topology and can be applied to any combination of mathematical functions.
  In addition, we represent an efficient way of doing statistical analysis for the protocol.
  Furthermore, we propose a novel, distributed and interactive algorithm to trace down the adversary and remove it from the network.   
  Finally, we do bandwidth analysis of the protocol and give the proof for the efficiency of the protocol.

\end{abstract}
