\chapter{June}

	\section{Interactive and non interactive proof}

	Interactive proof between two parties. 
	Setup, proof & probability. Probability of cheating is (1/2)^(#round).For example, for first round pr. of cheating is (1/2), for second round it is (1/4), for third round it is (1/8) and so forth.

	So, to have higher confidence you need to do more rounds which consumes a lot of bandwidth. So, the idea is to make it non interactive. 

	\section{zero knowledge proof}

	Magic of cryptography from Dan Bonneh.
	
	\section{commitment}
	Commitment schemes have important applications in a number of cryptographic protocols including secure coin flipping, zero-knowledge proofs, and secure computation.
	
	Interactions in a commitment scheme take place in two phases:
	the commit phase during which a value is chosen and specified the reveal phase during which the value is revealed and checked. In simple protocols, the commit phase consists of a single message from the sender to the receiver. This message is called the commitment. It is essential that the specific value chosen cannot be known by the receiver at that time (this is called the hiding property). A simple reveal phase would consist of a single message, the opening, from the sender to the receiver, followed by a check performed by the receiver. The value chosen during the commit phase must be the only one that the sender can compute and that validates during the reveal phase (this is called the binding property).
	
	\section{Meeting on 23rd June}

	- show what you know about cryptography.
	- 
