\chapter{Cheating Analysis}
	
	\begin{definition}
		A sensor node tampering with, the data-item to skew the final aggregate data-item or  off-path values to conceal its tampering activity and masquerading someone else as a cheater is consider as a \textbf{cheating}.	
	\end{definition}
	Because of the way aggregate commit algorithm works, an aggregate node has the highest power to do the cheating as described in Section \ref{sec:aggregate-adversary}. 
	The aggregate node gains more power to cheat as it climbs up in the aggregation tree.
	There are two potential phases where an aggregate node can cheat:
	\begin{itemize}
		\item While creating a commitment tree
		\item While distributing off-path values
	\end{itemize}
	We can detect a cheating activity with the help of the commitment field in the data-item as shown in Example \ref{ex1:cheating}.
	We want to identify the cheater and for 

	If an aggregate node cheats, it has to cheat at two phases in the protocol to conceal its cheating.
	First, it has to cheat while creating a commitment tree by changing one or more fields in the final aggregate data-item at the end of its aggregation.
	Secondly, it has to cheat while distributing the off-path values to conceal its cheating activity from its children.
	We show that because of the commitment field in the data-item it is impossible for an aggregate node to cheat without being detected.
	\section{Assumptions}
	We make following assumptions for the adversary.
		\begin{itemize}
			\item It does not tamper with the off-path data-items received from its parent.
			\item It can not send an authentication code with NACK message during verification of inclusion phase.
			\item It does not have the capability to masquerade by reproduce the signatures of any other sensor node.
		\end{itemize}
	Without these assumptions, there will be a lot of complainers in the network, creating a lot of traffic in the network.
	Ultimately, draining the battery levels of the sensor nodes until they die, making some sensor nodes in the network unreachable and potentially causing the denial-of-service attack in the network.
	% Because it can create many complainers in the network and network becomes unreachable, creating a Denial of service (DoS) attack in the network. 
	% % If it can tamper with the off-path data-items received from its parent than an adversary can mask someone else as a cheater.
	% If a cheater is allowed to send NACK message then it can send NACK messages all the time, creating a lot of traffic in the network which might create Denial of service attack.

	Following example shows the different ways an adversary can cheat in the smallest possible commitment tree and how the commitment field in the data-item can help us detect the cheating.
\begin{exmp}
	\label{ex1:cheating}
	Let's say the vertices in the commitment tree of Figure \ref{fig:cheating} have the data-items defined as follows.
	We did not include the signatures of these data-items as we assume that an aggregate node does not have the capability to reproduce the signatures of its childrens' data-items.\\
	\begin{figure}[t]
		\centering
		\includegraphics{images/commitment-tree-2.png}
		\caption{Smallest possible commitment tree}
		\label{fig:cheating}
	\end{figure}	
	$A_{0}$ = $<A_{id},1,10, H(N||1||10)>$\\
	$B_{0}$ = $<B_{id},1,20, H(N||1||20)>$\\
	$C_{1}$ = $<C_{id},2,30, H(N||2||30||A_{0}||B_{0})>$
	\begin{itemize}
	\item \textbf{No cheating}\\
		$C$ aggregates $B_{0},C_{0}$ according to the aggregate commit algorithm.\\
		\textbf{dissemination of root data-item}\\
			$A,B$ receives $C_{1}$ from the base station using authenticated broadcast.\\
		\textbf{dissemination of offpath values}\\
			$A$ receives $B_{0}$ from $C$ and vice versa.\\
		\textbf{verification of inclusion}\\
			$A_{0} + B_{0}$ = $<2,30,H(N||2||30||A_{0}||B_{0})>$ = $C_{1}$ (by A)\\
			$A_{0} + B_{0}$ = $<2,30,H(N||2||30||A_{0}||B_{0})>$ = $C_{1}$ (by B)
	\item \textbf{Cheating by replacing data-items}\\
		$C$ replaces $A_{0},B_{0}$ with $A'_{0},B'_{0}$ and then applies aggregate commit algorithm.\\
		$A'_{0}$ = $<A_{id},1,100, H(N||1||100)>$\\
		$B'_{0}$ = $<B_{id},1,200, H(N||1||200)>$\\
		$C'_{1}$ = $<C_{id},2,300, H(N||2||300||A'_{0}||B'_{0})>$\\
		\textbf{dissemination of root data-item}\\
			$A,B$ receives $C'_{1}$ from the base station using authenticated broadcast.\\
		\textbf{dissemination of offpath values}\\
			$A$ receives $B'_{0}$ from $C$ and vice versa.\\
		\textbf{verification of inclusion}\\
			$A_{0}+B'_{0}$ = $<2,210,H(N||2||210||A_{0}||B'_{0})>$ $\neq$ $C'_{1}$(by A)\\
			$A'_{0}+B_{0}$ = $<2,120,H(N||2||120||A'_{0}||B_{0})>$ $\neq$ $C'_{1}$(by B)
	\item \textbf{Cheating by tampering with data-items}\\
		$C$ tampers only with the value field in $A_{0},B_{0}$'s data-item and then applies aggregate commit algorithm.\\
		$A'_{0}$ = $<A_{id},1,100, H(N||1||10)>$\\
		$B'_{0}$ = $<B_{id},1,200, H(N||1||20)>$\\
		$C'_{1}$ = $<C_{id},2,300, H(N||2||300||A''_{0}||B_{0})>$\\
		$C''_{1}$ = $<C_{id},2,300, H(N||2||300||A_{0}||B''_{0})>$\\
		\textbf{dissemination of root data-item}\\
		$A,B$ receives $C'_{1}$ or $C''_{1}$ from the base station using authenticated broadcast.\\
		\textbf{dissemination of offpath values}\\
		$A$ receives $B''_{0}$ = $<B_{id},1,290,H(N||1||20)>$ from $C$\\
		$B$ receives $A''_{0}$ = $<A_{id},1,280,H(N||1||10)>$ from $C$\\
		\textbf{verification of inclusion}\\
		$A_{0}+B''_{0}$ = $<2,300,H(N||2||300||A_{0}||B''_{0})>\  \neq C'_{1} = C''_{1}$(by A)\\
		$A''_{0}+B_{0}$ = $<2,300,H(N||2||300||A''_{0}||B_{0})>\  = C'_{1} \neq C''_{1}$(by B)
	\item \textbf{Cheating by tampering with a single data-item}\\
		$C$ tampers $A_{0}$'s value field and then applies aggregate commit algorithm.\\
		$A'_{0}$ = $<A_{id},1,100, H(N||1||10)>$\\
		$C'_{1}$ = $<C_{id},2,120, H(N||2||120||A_{0}||B'_{0})>$\\
		$C$ creates $B'_{0}$ = $<B_{id},1,110, H(N||1||110)>$\\
		\textbf{dissemination of root data-item}\\
			$A,B$ receives $C'_{1}$ from $C$\\
		\textbf{dissemination of offpath values}\\
			$A$ receives $B'_{0}$ = $<B_{id},1,110, H(N||1||110)>$ from $C$\\
			$B$ receives $A_{0}$ = $<A_{id},1,10,H(N||1||10)>$ from $C$\\
		\textbf{verification of inclusion}\\
			$A_{0}+B'_{0}$ = $<2,120,H(N||2||120||A_{0}||B'_{0})>$ = $C'_{1}$ (by A)\\
			$A_{0}+B_{0}$ = $<2,30,H(N||2||30||A_{0}||B_{0})>$ $\neq$ $C'_{1} $ (by B)
	\end{itemize}
	\end{exmp}
Above example shows the significance of the commitment field in the data-item.
If an aggregator changes the value field in one of its children's data-item then to hide its misbehavior from its children, it has to compensate the difference with the relevant off-path data-item. 
% It means it has to use different data-item in the commitment.
\textbf{If an aggregation node has two unique children (not including itself) and if it tries to tamper with either one or both children's data-item then it can not create a fake data-item which will be accepted by both of its children.
One of it's children will complain in the verification phase as they will not be able to calculate the same root data-item received from the base station.}

\section{Possible Cheater Analysis}
	\begin{exmp}
		\begin{figure}[t]
			\centering
			\includegraphics{images/possible-cheaters.png}
			\caption{Smallest possible commitment tree}
			\label{fig:cheating}
		\end{figure}

		\begin{itemize}
			\item $A_{0}$ sends an authenticated code with NACK during verification of inclusion.
			Then possible adversaries are the following:
				\begin{itemize}
					\item $I$
					\item $(B, I)$
					\item $(B, M)$
					\item $(B, I, M)$
				\end{itemize}
		\end{itemize}
	\end{exmp}

% \section{What is not cheating ?}
	
% 	In figure 7.1, A is an aggregator if A is a cheater it can skew the final aggregation result irrespective of B's sensor reading. We do not consider this case as a cheating because A is adjusting its sensor reading, it's not changing the B's sensor reading. 
  
% 	For example, if maximum allowed value = 10\\
  
%   case I: $B_{0}(2)$ = 5, $A_{0}(2)$ = 13, $A_{1}(2)$ = 18. In verification, A will be caught due to out of range off path value.\\

%   case II: $B_{0}(2)$ = 5, $A_{0}(2)$ = 10, $A_{1}(2)$ = 15. $B_{0}^{'}(2)$ = 6, $A_{0}^{'}(2)$ = 9. that's not cheating.\\ 

% 	\begin{figure}[t]
% 		\centering
% 			\includegraphics[width=0.2\textwidth]{images/commitment_tree_1.png}\\
% 			\caption{Possible commitment tree}
% 	\end{figure}

% 	Similar arguments can be done for figure 7.2 if A, C  both are cheaters. In that case A is adjusting C's sensor reading to skew the final aggregation result and C will not complain as it is a cheater. We do not consider that as cheating either.

% \begin{figure}[t]
% 	\centering
% 		\includegraphics[width=0.2\textwidth]{images/commitment_tree_2.png}
% 		\caption{Possible commitment tree}
% 	\end{figure}

% \section{Probabilistic bound on a cheater}
	
% 	To derive Probabilistic bound on a cheater using following example.

% 	In figure 7.3, all vertices in a commitment tree are unique. And, remember cheater can not say NACK during verification phase. 

% 	\begin{itemize}
		
% 		\item $A_{0}$ says NACK during verification phase it implies that atleast one of the following is \{I\}, \{B, I\}, \{B, M\} is a cheater.
		
% 		\item $A_{0}, B_{0}$ says NACK during verification phase it implies that atleast one of the following is \{I\}, \{M\}, \{C, D, O\} is a cheater.

% 		\item $A_{0}, B_{0}, C_{0}$ says NACK during verification phase it implies that atleast one of the following is \{J ,I\}, \{J, M\}, \{D, O\} is a cheater.

% 		\item $A_{0}, B_{0}, C_{0}, D_{0}$ says NACK during verification phase it implies that atleast one of the following is \{O\}, \{M\}, \{I, J\}, \{E, F, G, H, O\} is a cheater.

% 		\item $A_{0}, B_{0}, C_{0}, D_{0}, E_{0}$ says NACK during verification phase it implies that atleast one of the following is \{O, K\}, \{M, K\}, \{I, J, K\}, \{F, G, H, O\} is a cheater.

% 		\item $A_{0}, B_{0}, C_{0}, D_{0}, E_{0}, F_{0}$ says NACK during verification phase it implies that atleast one of the following is \{I, J, K\}, \{M, N\}, \{O, K\}, \{O, N\} is a cheater.

% 		\item $A_{0}, C_{0}$ says NACK during verification phase it implies that atleast one of the following is \{I\}, \{J\} is a cheater.

% 	\end{itemize}	

% 	\begin{figure}[t]
% 		\centering
% 			\includegraphics[width=0.7\textwidth]{images/commitment_tree_3.png}
% 			\caption{Possible commitment tree}
% 	\end{figure}

% 	Similar, kind of analysis can be done for figure 7.4 in which all the vertices in the commitment tree are different. 

% 	\begin{figure}[t]
% 		\centering
% 			\includegraphics[width=0.7\textwidth]{images/commitment_tree_4.png}
% 			\caption{Possible commitment tree}
% 	\end{figure}

% 	From all above examples we can derive the following pattern as well,

% 	If d = depth of a tree,\\

% 	\begin{tabular}{| l | l |}
%     \hline
%     Depth of a cheater & Minimum number of NACK messages \\ \hline
%     d - 1 & 1 \\ \hline
%     d - 2 & 2 \\ \hline
%     d - 3 & 4 \\ \hline
%     d - 4 & 8 \\ \hline
%   \end{tabular}


% 	\section{ Why do we need digital signatures ?}
% 	Digital signatures allow us to achieve authenticity of the message. 
% 	The labels and signatures have the following format:

% 	id = id

% 	label = $<count, value, commitment>$

% 	signature = $E_{Private_key}( H( N || label ) )$\\
% 	Where \textit{count} is the number of leaf vertices in the subtree rooted at this vertex; 
% 	\textit{value} is the SUM aggregate computed over all the leaves in the subtree; \textit{id} is the sum of all the leaves id in the subtree; \textit{signature} is a cryptographic scheme for demonstrating the authenticity of a message; \textit{N} is the query nonce. 
	
% 	There is one leaf vertex $u_{s}$ for each sensor node s, which we call the leaf vertex of s. The label of $u_{s}$ consists of count = 1, value = $a_{s}$ where $a_{s}$ is the data value of s, and signature is the node's unique ID.

% 	Internal vertices represent aggregattion operations, and have labels that are defined based on their children. Write up examples after talking to Dr.King : Do you have to aggregate ID's as well ?

% 	\section{ Why digital signatures are not sufficient to detect a cheater ? or Why do we need public key infrastructure to detect a cheater ?}
% 	Digital signatures allow us to achieve authenticity of the message but do not provide any mechanism to achieve integrity of the message. To achieve integrity we need public key infrastructure. 

% 	For example, in figure 7.3 one set of possible lables could be the following:

% 	$id_{A} = 1; A_{0} = <1, 5, H(N||1||5)>; Sig A_{0} = E_{K_{A}}(H(N||A_{0})); $

% 	$id_{B} = 2; B_{0} = <1, 6, H(N||1||5)>; Sig B_{0} = E_{K_{B}}(H(N||B_{0})); $

% 	$id_{I} = 3; I_{1} = <2, 11, H(N||2||11||A_{0}||B_{0})>; Sig I_{1} = E_{K_{I}}(H(N||I_{1})); $

% 	$id_{J} = 4; J_{1} = <2, 15, H(N||2||15||C_{0}||D_{0})>; Sig J_{1} = E_{K_{J}}(H(N||J_{1})); $

% 	$id_{M} = 5; M_{2} = <4, 26, H(N||4||26||I_{1}||J_{1})>; Sig M_{2} = E_{K_{M}}(H(N||M_{2})); $

% 	Above labels and signatures are the case where no one is cheating in the network. If A, B say NACK message during the verification phase it means either M or I is a cheater. To preceisely find who is cheater we have following problems:

% 	\begin{itemize}
% 		\item M can say it received ($I_{1}^{'}, Sig I_{1}$) eventhough it received ($I_{1}, Sig I_{1}$) from I.
% 		\item M can not verify that it received ($I_{1}^{'}, Sig I_{1}$) instead of ($I_{1}, Sig I_{1}$) from I.
% 	\end{itemize}
% 	Because of this we can not not detect cheater between I, M. The fundamental problem is that signatures can be verified only by the base station and not by any of the intermediate nodes. We want the ability in which an intermediate node can verify the signatures from its children. And that is why we need public key infrastructure.

\section{Random thoughts on cheating}
 We know that there are two potential phases where an aggregate node can cheat.
 But if an aggregate node cheats during the dissemeination of off-path values that is cheating but an adversary is not gaining anything except the fact that it is creating unnecessary traffic in the network.
 The goal of an adversary to brake a secure aggregation algorithm is to make the base station believe the out of range aggregation result without being detected.
 By tampering with the off-path values it can neither hide it self from its tampering of data-items nor it can make base station believe the out of range aggregation result. 
 Hence, we assume that an aggregate node does not tamper while distributing off-path values.
 
 For an aggregation algorithm to be secure