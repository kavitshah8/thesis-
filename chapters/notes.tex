\chapter{Notes}

\begin{itemize}
	\item Do we hash node's id in the commitment field of its data-item? NO
	\item Do we send the payload signature in case of Figure 5.4? YES
\end{itemize}

To do list:
\begin{itemize}
	% \item ch-3: signatures from 628; All PKI are not digital signatures capable.
	% \item ch-5: Clear goals of the protocol.
	% \item Review the writing.
	% \item ch-2: Add smooth transition to the security section.
	% \item ch-2: add more applications in section 2.1.
	% \item ch-4: change sign label
	% \item ch-4: Why SHIA does not have ID in their label? Why do you have ID? What benefits you get out of it?
	% \item Cheating: Improve existing example.
	% \item Cheating: Analysis of possible cheaters.
	\item Meeting: 
	\begin{itemize}
	\item short names for two approaches.
	\item MACs vs Signatures in collection of Authentication codes
	\end{itemize}
	\item ch-3: Digital signatures helps authentication, non-repudiation, data integrity
	\item Cheating: Cheating detection with CT generation.
	\item Cheating: Cheating detection with off-path distribution.
	\item Cheating: Cheating detection with both scenarios in place.
	\item Fix images with Power point.
	\item in-network, data-item, transmit-payload, off-path

\end{itemize}
	
	Signature on the payload is important for signifying not cutting the tree.

	Compare number of signatures in both approaches are same but amount of signing is different.
	Compare number of certificates in both approaches.

	Matrix for choice of protocol
		Number of certificates
		Number of signing
		Number of signatures
		Power consumption

	Matrix for different topologies
		star, palm, binary 
		
	Include verification with the protocol.

	Savings the BW then make decision with power analysis.

We should meet when what to write?