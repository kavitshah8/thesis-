\chapter{Secure Hierarchical In-network data aggregation} % (fold)
\label{cha:Secure Hierarchical In-network data aggregation}

We describe the Secure Hierarchical In-network data aggregation $\textit{(SHIA)}$ protocol of \cite{chan2006secure} as our work enhances this protocol by making it more efficient and adding new capabilities to the protocol. The goal of $\textit{SHIA}$ is to compute aggregate functions (such as $\textit{SUM}$, $\textit{AVERAGE}$, $\textit{COUNT}$) of the sensed values by the sensor nodes while assuming that a portion of the sensor nodes are controlled by an adversary which is attempting to skew the final result.

\section{Network Assumptions}
\section{Security Infrastructure}
\section{Attacker Model}
\section{Problem Definition}
\section{The SUM Aggregate Algorithm}
	In this algorithm, the aggregate function $f$\ is addition meaning that we want to compute $a_{1} + a_{2} + \dotsc + a_{n}$, where $a_{i}$\ is the sensed data value of the node $i$.
	This algorithm has three main phases:
	\begin{itemize}
		\item Query dissemination
		\item Aggregate commit
		\item Result checking
	\end{itemize}

	\subsection{Query dissemination}
		Prior to this phase an aggregation trees is created using a tree generation algorithm.
		We can use any tree generation algorithm as this protocol works on any aggregation tree structure.
		For completeness of this protocol, one can use Tiny Aggregation Service (TaG)\cite{madden2002tag}.
		TaG uses broadcast message from the base station to initiate a tree generation.
		Each node selects its parent from whichever node it first receives the tree formation message.  

		To initiate the query dissemination phase, the base station broadcasts the query request message with the query nonce $N$\ in the aggregation tree. 
		The query request message contains new query nonce $N$\ for each query to prevent replay attacks in the network.
		It is very important that the same nonce is never re-used by the base station.
		$SHIA$\ uses \textbf{hash chain} to generate new nonce for each query. 
		A hash chain is constructed by repeatedly evaluating a pre-image resistant hash function $h$\ on some initial random value, the final value (or ``anchor value'') is preloaded on the nodes in the network.
		The base station uses the pre-image of the last used  value as the nonce for the next broadcast.
		A hash chain prevents an adversary from predicting the query nonce for future queries as it has to reverse the hash chain computation to get an acceptable pre-image.

	\subsection{Aggregate commit} % (fold)
	\label{sub:aggregate_commit}
	
	% subsection aggregate_commit (end)

Then elaborate your approach.
Two differences:
	data-item format
	CT generation being root in as many trees as possible