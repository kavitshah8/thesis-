\chapter{Verification}


\section{dissemination final commitment}
\section{dissemination of off-path values}
	Two cases:
	\begin{itemize}
		\item {With signatures}
		\item {Without signatures}
	\end{itemize}
\section{verification of inclusion}
\section{collection of authentication codes}
\section{verification of authentication codes}
	The authentication codes for sensor node $s$, with either positive or negative acknowledgment message, are defined as follows:
	\begin{equation}
		MAC_{K_{s}}(N\ ||\ \textit{ACK})
	\end{equation}
	\begin{equation}
		MAC_{K_{s}}(N\ ||\ \textit{NACK})
	\end{equation}
	$K_{s}$ is the key that $s$ shares with the base station;
	$\textit{ACK}$, $\textit{NACK}$ are special messages for positive and negative acknowledgment respectively.
	The authentication code with $\textit{ACK}$ message is sent by the sensor node if it verifies its contribution correctly to the root commitment value during the 
	\textit{verification of inclusion} phase and vice versa.
	
	To verify that every sensor node has sent its authentication code with \ack, the base station computes the $\Delta_{ack}$ as follows:
	\begin{equation}
		\displaystyle{\Delta_{ack} = \bigoplus_{i = 1}^n MAC_{K_{i}}(N || ACK) }
	\end{equation}
	The base station can compute $\Delta_{ack}$ as it knows $K_{s}$\ for each sensor node $s$.
	Then it compares the computed $\Delta_{ack}$\ with the received root authentication code $\Delta_{root}$\ from the root of the aggregation tree. 
	If those two codes match then it accepts the aggregated value or else it proceeds further to find an adversary. 

	To detect an adversary, the base station needs to identify which nodes in the aggregation tree sent its authentication codes with \nack\ during the verification of inclusion phase.
	The node who sent authentication code with \nack\ during the verification of inclusion phase is called a \complainer. 
	We claim that if there is a single complainer in the aggregation tree during the verification of inclusion phase then the base station can find the complainer in linear time.
	To find a complainer, the base station computes the complainer code $c$.
	\begin{equation}\label{eq:complainer}
		c := \Delta_{root} \oplus \Delta_{ack}
	\end{equation}
	Then it computes the complainer code $c_{i}$\ for all node $i = 1, 2, \dotsc, n$. 
	\begin{equation}\label{eq:caomplainer-node}
		c_{i} := MAC_{K_{i}}(N\ ||\ \textit{ACK}) \oplus MAC_{K_{i}}(N\ ||\ \textit{NACK})
	\end{equation}
	Then it compares $c$\ with all $c_{i}$\ one at a time. 
	The matching code indicates the complainer node.
	The base station needs to do $n \choose 1$\ calculations according to Equation \ref{eq:caomplainer-node} and same number of comparisons to find a complainer in the aggregation tree. 
	Hence, the base station can find a single complainer in linear time.
	\begin{exmp} If there are four nodes ${s_{1},s_{2},s_{3},s_{4}}$ in an aggregation tree and their authentication codes with \ack, \nack\ message in the binary format are defined below.\\
		$\mac_{K_{1}}(N\ ||\ \ack)$ = $(1001)_{2}\ ;\ $
		$\mac_{K_{1}}(N\ ||\ \nack)$ = $(1101)_{2}$\\
		$\mac_{K_{2}}(N\ ||\ \ack)$ = $(0110)_{2}\ ;\ $
		$\mac_{K_{2}}(N\ ||\ \nack)$ = $(1111)_{2}$\\	
		$\mac_{K_{3}}(N\ ||\ \ack)$ = $(0101)_{2}\ ;\ $
		$\mac_{K_{3}}(N\ ||\ \nack)$ = $(0111)_{2}$\\
		$\mac_{K_{4}}(N\ ||\ \ack)$ = $(0011)_{2}\ ;\ $
		$\mac_{K_{4}}(N\ ||\ \nack)$ = $(1110)_{2}$\\
		$\Delta_{root} = (0100)_{2}$\\
		$\Delta_{ack} = (1101)_{2}$\\
		$c_{1} = (0100)_{2}$, $c_{2} = (1001)_{2}$, $c_{3} = (0010)_{2}$, $c_{4} = (1101)_{2}$\ \\
		$c = (1101)_{2}$
		$c$ is equal to $c_{4}.$\\
		So, the base station identifies that the $s_{4}$\ complained, during verification of inclusion phase.\\ 
	\end{exmp}
	In general, to find $k$\ complainers the base station needs to do $ n \choose k$\ calculations and the same number of comparisons to find $k$\ complainers.
	
	\textcolor{red}{How XOR is negating the contribution of NACK.}
	\[ 
		\left( 
			\begin{array}{cccc}
				1 & 0 & 0 & 1 \\ 
				0 & 1 & 1 & 0 \\
				0 & 1 & 0 & 1 \\
				0 & 0 & 1 & 1 \\
				\hline
				1 & 0 & 0 & 1 
			\end{array}
		\right)
	%
		\left( 
			\begin{array}{cccc}
				1 & 1 & 0 & 1 \\ 
				1 & 1 & 1 & 1 \\
				0 & 1 & 1 & 1 \\
				1 & 1 & 1 & 0 \\
				\hline
				1 & 0 & 1 & 1 
			\end{array}
		\right)
	\]

	The base station receives the following:
	\[ 
		\left( 
			\begin{array}{cccc}
				1 & 0 & 0 & 1 \\ 
				0 & 1 & 1 & 0 \\
				0 & 1 & 0 & 1 \\
				1 & 1 & 1 & 0 \\
				\hline
				0 & 1 & 0 & 0 
			\end{array}
		\right)
	\]

	The base station does the following:

	\[
		\left( 
			\begin{array}{cccc cccc cccc cccc}
				1 & 0 & 0 & 1\ \vline\  0 & 1 & 1 & 0\ \vline\  0 & 1 & 0 & 1\ \vline\  0 & 0 & 1 & 1 \\
				1 & 1 & 0 & 1\ \vline\  1 & 1 & 1 & 1\ \vline\	0 & 1 & 1 & 1\ \vline\	1 & 1 & 1 & 0 \\ 
				\hline
				0 & 1 & 0 & 0\ \vline\ 1 & 0 & 0 & 1\ \vline\ 0 & 0 & 1 & 0\ \vline\ 1 & 1 & 0 & 1\\
			\end{array}
		\right)
	\]

	\[ 
		\left( 
			\begin{array}{cccc}
				1 & 0 & 0 & 1 \\ 
				0 & 1 & 0 & 0 \\
				\hline
				1 & 1 & 0 & 1 \\
			\end{array}
		\right)
	\]

	And concludes that node $4 $ is complaining.
\newpage
\section{Detect an adversary}
\begin{algorithm}
\caption{Pseudo algorithm to detect an adversary}

	\begin{algorithmic}[1]

			\STATE $BS$ \ identifies all the complainer and creates $c = \{c_{1}, c_{2}, \dotsc, c_{n}\}$
			\FORALL {$N \in c$}

				\STATE $BS$\ asks $N$ to send data-items with its signature, sent during commitment tree generation phase
			
			\ENDFOR

			\STATE $BS$\ identifies possible adversary based on $c$ and creates $a = \{a_{1},a_{2},\dotsc,a_{n}\}$

			\FORALL {$A \in a$}

				\STATE $BS$\ asks $A$ to send data-items with its signature, received and sent by $A$ during commitment tree generation phase
				\STATE If needed $BS$\  asks the parent of $A$ to send data-items with its signature
	
			\ENDFOR

			\STATE $BS$\ determines the adversary based on the verification of signatures

	\end{algorithmic}
\end{algorithm}

\begin{theorem}
	%\emph{(Lagrange's Theorem)}
	\label{Commitment tree}
	Binary commitment tree is optimal in terms of verification as it requires minimum number of off-path values.
\end{theorem}

\begin{proof}
	Let us say $n$ is the number of leaves in the given commitment tree.

	$ \log _3( n ) = y $

	$ 3^y = n $

	$ \log_2( 3^y ) = \log_2( n ) $

	$ y * \log_2( 3 ) = \log_2( n ) $

	$ \log_3( n )*\log_2( 3 ) = \log_2( n ) $

	$ \log_3( n ) = \frac{ {\log _2 ( n )} }{{\log _2 ( 3 )}} $

	$ 2 * \log_3( n ) = [2 / \log_2( 3 ) ]* log_2( n ) = ( 1.2618 ) * log_2( n ) $

	$ 2 * log_3( n ) > log_2( n ) $ \\
	For the given binary commitment tree, each leaf vertex needs $\log_{2}(n)$ off-path values in the verification phase.
	The total off-path values needed in the given commitment tree is $n \cdot \log_{2}(n)$.\\
	For the given tertiary commitment tree, each leaf vertex needs $2 \cdot \log_{3}(n)$ off-path values in the verification phase.
	The total off-path values needed in given commitment tree is $2 \cdot n \cdot \log_{3}(n)$.

	Hence, in totality the binary commitment tree requires the minimum number of off-path values.
\end{proof}