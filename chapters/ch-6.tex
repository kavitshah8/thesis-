\chapter{SYSTEM DESIGN} % (folit
\label{cha:System Design}
	
	This chapter describes the rationale behind the design specifications of our secure aggregation protocol for sensor networks, the mechanism used to achieve it and implementation details.

	\section{Introduction}

	The most significant design aspect of the sensor network is the \textit{Lifetime of the Network}.
	A sensor network tends to have limited life span as they are powered by the battery.
	The lifetime of the sensor network is inversely proportional to the sensor nodes' power consumption.
	One of the most dominating factor for the power consumption is transmitting and receiving data between sensor nodes, making network bandwidth an expensive resource.
	The bandwidth is a more expensive resource than the local data computation, as trans-receiving activity consumes more power than computation.
	The obvious solution to increase the lifespan of the sensor network is to decrease the bandwidth usage in the network.
	As we know, data aggregation techniques can greatly reduce the bandwidth usage in the network, increasing the lifespan of the network.
	Hence, data-aggregation techniques are one of the key tool in our tool box while designing protocol for the sensor networks.

	The second most significant factor in the design of the data aggregation protocol is \textit{Security Architecture }.
	The International Telecommunications Union (ITU) Telecommunication Standardization Sector (ITU-T) Recommendation X.800, \textit{Security Architect for Open Systems Interconnection} (OSI) provides a systematic approach for it.
	The OSI security architecture focuses on security attacks, mechanism and services defined as follows:
	\begin{description}
		\item[Security attack] is any action that compromises the security of information owned by an organization or entity \cite{stallings2008computer}.
		\item[Security mechanism] is a mechanism that is designed to detect, prevent, or recover from a security attack
		\cite{stallings2008computer}.
		\item[Security service] is a service that enhances the security of the data processing systems and the information transfers of an organization. The services are intended to counter security attacks, by using one or more security mechanisms \cite{stallings2008computer}.

	\end{description}

	Security attacks can be classified as \textit{active attacks} and \textit{passive attacks}.
	
	Active attacks involve some modification of the data stream or the creating of a false stream and can be subdivided into four categories: \textit{replay, masquerade, modification of messages, and denial of service}.
	Replay involves the passive capture of a data unit and its subsequent retransmission to produce an unauthorized effect.
	A masquerade takes place when one entity pretends to be a different entity.
	For example, authentication sequence can be captured and replayed after a valid authentication sequence has taken place, thus enabling an authorized entity with few privileges to obtain extra privileges by impersonating an entity that has those privileges.
	Modification of message means that some portion of a legitimate message is altered or that the messages are delayed or reordered, to produce an unauthorized effect.
	For example, a message stating ``Allow Barack Obama to read confidential file accounts.'' is modified to say ``Allow Michelle Obama to read confidential file accounts.''
	The denial of service inhibits the normal use of communication facilities.
	For example, an entity may suppress all messages directed to a particular destination.
	Another form of service denial is the disruption of an entire network, either by disabling the network or by overloading it with messages so as to degrade performances.
	Active attacks present the opposite characteristics of passive attacks.
	Whereas passive attacks are difficult to detect, measures are available to prevent their success.
	On the other hand, it is quite difficult to prevent active attacks absolutely, because to do so would require physical protection of all communication facilities and paths at all times.
	Instead, the goal is to detect them and to recover from any disruption or delays caused by them.
	Because the detection has a deterrent effect, it may also contribute to prevention.
	
	Passive attacks attempts to learn or make use of information from the system but does not affect the system resources.
	They are in the nature of eavesdropping on, or monitoring of, transmissions.
	% Two types of passive attacks are \textit{release of message contents} and \textit{traffic analysis}.
	% The \textit{release of message contents} is well understood. 
	% A telephone conversation, an electronic mail, and a transferred file may contain sensitive or confidential information.
	% We would like to prevent an opponent from learning the contents of these transmissions.
	% The \textit{traffic analysis} is subtler. 
	% Suppose that we had a way of masking the message so that even if the  opponents captured the message, could not extract it.
	% The common mechanism for masking the message is encryption.
	% If we had encryption protection in place, an opponent might still be able to observe the pattern of these message communication.
	% The opponent could determine the location and identity of communicating hosts and could observe the frequency and length of message being exchanged.
	% These patterns might be useful in guessing the nature of the communication that was taking place.
	Passive attacks are difficult to detect as they do not involve any alteration of the message.
	Typically, the message traffic is sent and received in an apparently normal way and neither the sender nor receiver is aware that a third party has sniffed the message or observed the traffic pattern.
	However, it is feasible to prevent the success of these attacks, usually by means of encryption. 
	Thus, the emphasis on dealing with passive attacks is on prevention rather that detection.
	Passive attacks are subtler and outside the scope of our thesis.

	Security services are divided into six categories:
	\textit{Authentication, Access Control, Data Confidentiality, Data Integrity, Non-repudiation, Availability}.
	
	Authentication service assures that a communication is authenticate.
	Two types of authentication services are described in the standard: \textit{peer entity authentication} and \textit{data origin authentication}.
	The peer entity authentication provides for the corroboration of the identity of a peer entity in an association.
	Two entities are considered as peer if they implement the same protocol in different systems (e.g., two TCP users in two communicating systems).
	Peer entity authentication is used at the establishment of, or at times during the data transfer phase of, a connection.
	It attempts to provide confidence that an entity is not performing either a masquerade or an unauthorized replay of a previous connection.
	Data origin authentication provides the corroboration of the source of a data unit. 
	It does not provide protection against the duplication or modification of data units.
	This type of service supports applications like electronic mail where there are no prior interactions between the communicating entities.
	
	Data Confidentiality ensures the protection of transmitted data from passive attacks.
	It is also known as secrecy or privacy.
	For example, student grade information is an asset whose confidentiality is considered to be highly important by students.
	In the United States, the release of such information is regulated by the Family Educational Rights and Privacy Act (FERPA).
	Grade information should only be available to students, their parents, and employees that require the information to do their job.
	The other aspect of confidentiality is the protection of traffic flow from analysis.
	This requires that an attacker not be able to observe the source and destination, frequency, length, or other characteristics of the traffic on a communications facility.
	Ensuring confidentiality can be difficult.
	For example, who determines which parties are authorized to access the information?
	By ``accessing'' information, do we mean that an authorized party can access a single bit? the whole collection? pieces of information out of context?
	Can someone who is authorized disclose that information to other parties?\cite{pfleeger2002security}
	
	Data Integrity assures that the only authorized entities can modify the information; where modification means writing, creating, deleting and changing the information.
	A hospital patient's allergy information stored in a database illustrates the several aspects of the data integrity.
	The doctor should be able to trust that the information is correct and current.
	Now, suppose a nurse who is authorized to view and update the information deliberately falsifies the data to cause harm to the hospital.
	The database needs to be restored to a trusted basis quickly, and it should be possible to trace the error back to the person responsible.
	Patient allergy information is an example of an asset with a high requirement for integrity. 
	Inaccurate information could result in serious harm or death to a patient and expose the hospital to massive liability.
	Because the integrity service relates to active attacks, we are concerned with detection rather than prevention.
	If a violation of integrity is detected, then the service may simply report this violation, and some other portion of software or human intervention is required to recover from the violation.
	Alternatively, there are mechanisms available to recover from the loss of integrity of data, as we will review subsequently. The incorporation of automated recovery mechanisms is, in general, the more attractive alternative.
	
	Availability  means that the information of a system being accessible and usable by an authorized parties according to performance specification at appropriate times.
	In other words, if some person or system has legitimate access to a particular set of objects, that access should not be prevented.
	Availability applies to the information and services.
	For example, information or service is available can mean the following. 
	It is present in the usable form, having the enough capacity to meet the service need.
	If it is in wait mode, it is making enough progress, and it has a bounded waiting time, meaning there is timely response to our requests.
	Resources are allocated fairly so that some requests are not favored over the others.
	The service can be used easily and in its intended way.
	Availability addresses the concern raised by the denial of service attacks.
	It depends on proper management and control of system resources and thus depends on access control service and other security services.
	
	Non repudiation service requires neither the sender nor the receiver can deny the transmission.
	Thus, when a message is sent, the receiver can prove that the alleged sender in fact sent the message.
	Similarly, when a message is received, the sender can prove that the alleged receiver in fact received the message.
	This is very important in electronic commerce applications, where it is important that a consumer cannot deny the authorization of a purchase.
	
	Access Control is the ability to restrict and control the access to host systems and applications via communications links.
	To achieve this, each entity trying to gain access must first be identified, or authenticated, so that access rights can be tailored to the individual.
	Login credentials and locks are two analogues mechanism of access control.

	Security mechanisms follow one of the most fundamental principle of Open Design \cite{bishop2004introduction} defined as follows:
	\begin{definition}
		The principle of open design states that the security of a mechanism should not depend on the secrecy of its design or implementation.
		\label{def:open-design}
	\end{definition}
	Designers and implementers of a program must not depend on secrecy of the details of their design and implementation to ensure security.
	Others can ferret out such details either through technical means, such as disassembly and analysis, or through nontechnical means, such as searching through garbage receptacles for source code listings (called ``dumpster-diving'').
	If the strength of the program's security depends on the ignorance of the user, a knowledgeable user can defeat that security mechanism.
	The term ``security through obscurity'' captures this concept exactly.
	This is true of cryptographic software and systems.
	Because cryptography is a highly mathematical subject, and companies who sell these softwares want to keep their algorithms secret. 
	Issues of proprietary software and trade secrets complicate the application of this principle.
	In some cases, companies may not want their designs made public, lest their competitors use them.
	The principle then requires that the design and implementation be available to people barred from disclosing tit outside the company.
	Note that keeping cryptographic keys and passwords does not violate this principle, because a key is not an algorithm.
	However, keeping the enciphering and deciphering algorithms secret would violate it.
	The following security mechanisms may be incorporated into the appropriate protocol layer in order to provide some of the OSI services.
	\textit{Encryption} is the use of mathematical algorithms to transform data into a form that is not readily intelligible.
	The transformation and subsequent recovery of the data depend on an algorithm and zero or more encryption.
	\textit{Digital Signature} appends the data to a cryptographic transformation of data unit that allows a recipient of the data unit to prove the source and integrity of the data unit and protects against the forgery (e.g., by the recipient).
	\textit{Hash functions} provides one way functions to achieve irreversible encryption.
	\textit{Traffic Padding}, \textit{Routing Control}, \textit{Notarization}, \textit{Authentication Exchange} are possible mechanisms as well but they are outside the scope of our thesis.

	% \textit{Traffic Padding} is the insertion of bits into gaps in a data stream to frustrate traffic analysis attempts.

	% \textit{Routing Control} enables selection of particular physically secure routes for certain data and allows routing changes, especially when a breach of security is suspected.

	% \textit{Notarization} is the use of a trusted third party to assure certain properties of a data exchange.

	% \textit{Authentication Exchange} is a mechanism intended to ensure the identity of an entity by means of information exchange.

	As you can see, expectations from secure networking protocol are far-reaching.
	It is very ambitious for any system to have all the above mentioned security services at the same time.
	If we have to achieve all the security services for sensor networks we can make each sensor node signs its reading data and then send data with its signature to its parent.
	The network forwards all the data with their respective signatures to the base station.
	The base station verifies all the signatures and then calculates the aggregate function.
	Clearly, this approach is not practical as it requires $n$ signatures to traverse through the link between the base station and the root node of the network; where $n$ is the number of nodes in the network.
	Which makes that link the bottleneck link of the network. 
	And if that link breaks we loose the entire network connectivity.

	In reality all the security services are not always required.
	For example, Protecting banner advertisements on web pages. 
	The provider of the advertisements do not care if they copy the advertisements and show it to other people.
	So, there is no confidentiality at all.
	But they want to prevent people from changing the advertisements to the different types of advertisements.
	Also, when a client downloads a file from the file server using Internet, he can verify the integrity of the file using the checksum.
	But it is okay if somebody on the network sniffed the downloading activity as far as it did not change the content of the file.
	In this application, a client requires the integrity of the file but the privacy of the client, the file server and the Internet service provider are not required.
	To give an analogy with a physical world application, when a person writes a postcard, he puts his own signature on the it. 
	When that postcard delivers successfully by the postal service, the receiving party can verify the integrity of the message from the handwriting and the signature of the person.
	The postal service knows the sender and the receiver of the postcard.
	It can even read the postcard.
	Again, the integrity is important not the confidentiality.
	In most engineering discipline, it is useful to clarify the requirements carefully before embarking on a project \cite{2002-Stajano-ubiquitous}.
	An important aspect to the computer security, as discussed by Anderson \cite{anderson1993cryptosystems}, that is often ignored: designed the security before careful thought of the needs.
 	It is crucial to find out which security services are desired for the particular application, so we can use the relevant security mechanisms to provide those services.
	
	% In practice, protocol designer finds out the most important security services for the applications before designing the protocol.

	For example, Wagner's work \cite{wagner2004resilient} describes the attacks on standard schemes for data aggregation and introduces the problem of securing aggregation in the presence of malicious or spoofed data.
	He proposes a mathematical theory of security for aggregation.
	The theory quantifies, in a principled way, the robustness of an aggregation operator against malicious data.
	It draws novel connections to statistical estimation theory and to the field of robust statistics.
	He identifies techniques for aggregation that provide robustness against attack. 
	It provides helpful guidance to sensor network implementors or designers in selecting appropriate aggregate functions.

	Secure Information Aggregation (SIA) \cite{przydatek2003sia}  addresses the problem of how to enable secure information aggregation, such that the user accepts the data with high probability if the aggregated result is within a desired bound, but that the user detects cheating with high probability and rejects the result if it is outside of the bound.
	SIA provides statistical security under the assumption of a single-aggregator model.
	In the single-aggregator model, sensor nodes send their data to a single
	aggregator node, which computes the aggregate and sends it to the
	base station.
	This form of aggregation reduces communications only on the link between the aggregator and the base station, and is not scalable to large multihop sensor deployments.
	SIA provides the probabilistic data-integrity service under strong network assumptions.

	Secure Hierarchical In-network Aggregation (SHIA) \cite{chan2006secure}, in many ways enhances Secure Information Aggregation (SIA).
	SHIA presents the provably secure sensor network data aggregation protocol for general networks and multiple adversarial nodes, in compare to SIA which provides probabilistic security for a single aggregate network topology.
	SHIA limits the adversary’s ability to manipulate the aggregation result with the tightest bound possible, with no knowledge of the distribution of sensor data values.
	SHIA provides data-integrity service for any hierarchical sensor networks because of that it can detect malicious activity in the network.

	The third most significant aspect is \textit{Data Integrity}.
	% The data collected from the sensors gets aggregated on the way to the base station.
	% The base station takes an important decision based on the received value.
	Suppose, the sensor network is deployed to measure the certain harmful chemical levels in the lab atmosphere and raise an alarm if the it increases above certain level.
	If the base station fails to raise an alarm because of the false aggregated data, can create catastrophic and lethal situation.
	Hence, the data integrity is so important.
	
	Data integrity can be achieved with the error detection and error correction techniques.
	The first step towards achieving the data integrity in the sensor network is to \textit{detect any malicious activity} in the network.
	The second step is the ability to \textit{locate the malicious node or an adversary} in the network.
	We think detecting a malicious activity without tracing down the malicious node responsible for it, is of no value.
	To give an analogy with the physical world, it is like you know there is a crime in the city and you do not do anything about it.
	Locating the criminals responsible for the crime is mandatory to abolish the crime in the city.
	In similar way, locating the malicious node and removing it from the network is an important security service. 
	Failing to provide such a security service, allows an adversary to continue doing the malicious activity in the future, which makes aggregated sensor data garbage. 
	The network has to redo the all the work to create a response for the query from the base station, which consumes lots of bandwidth. 
	An adversary can repeat this process until the network dies due to low battery power, creating a denial of service attack in the network.
	Hence, detecting a malicious node who is responsible for the malicious activity is equally important.
	If we can track down the malicious node in the network then we can remove that node from the network for all future queries.
	And make sure that all the future queries are not manipulated by any malicious node in the network.
	Hence, the fourth and fifth most significant design aspects of secure aggregation protocol are \textit{detect the malicious activity} (in terms of data aggregation) and can \textit{detect the adversary} in the network, respectively.

	To detect an adversary (or prove someone guilty), we need proof that the adversary is responsible for the malicious activity.	 
	% To achieve our goal, we think the Digital signature is the perfect cryptographic tool.
	Consider the following example showing the analogy with the signature scheme in the physical world, used by the postal services.
	When a postman delivers the package, the receiving party has to sign the document informing that he verified and received the package.
	Since only the receiving party can create his signature, in the future he can not claim of not receiving the package or receiving the damaged or incorrect package. 
	If he claims such, the postal company has the signed document as the proof mentioning that the package was received successfully, by the receiving party.
	The signed document also ensures to the postal company that the postman did not misplace or steal the package.
	The signature scheme used by the postal service promises non repudiation security service.
	% authenticity (assures the origin of the message) and 
	Hence, we require \textit{non-repudiation} security service in the sensor network.

	% With Digital signatures one can achieve \textbf{non-deniability  (non repudiation), authentication, integrity} (assures that only the authorized parties can modify the message) as described in Section \ref{sec:digital-signature}.

\section{System Design Specifications}
	We want to design a secure aggregation protocol which maximizes the \textit{network lifetime}.
	The protocol can be applied to \textit{any hierarchical sensor network}.
	Further, we want protocol to work on \textit{resilient} and \textit{non-resilient} aggregate functions without compromising any desired security properties.
	In addition, We want protocol to be secure with \textit{a single} or \textit{multiple adversaries} in the network.
	Moreover, we want the protocol to protect against any \textit{active attacks}. 
	If the aggregation result is accepted by the base station it should have very high confidence in the result, meaning that we want the highest level of \textit{data integrity} security service in the protocol.
	We need the capabilities to \textit{detect any malicious activity}.
	If there is any malicious activity in the network, we should be able to \textit{locate an adversary} responsible for it, in the network.
	Also, we want the \textit{non-repudiation} security service so neither sender nor receiver can deny the transmission or receiving of the message, which is mandatory to locate an adversary.
	Finally, we want to achieve this with the least amount of \textit{bandwidth and computation} overhead in the network.
	So, the protocol can be easily implemented in the real world sensor networks.
	It is not that difficult to provide mentioned security properties to protect against active attacks in the sensor network. 
	We will show that it requires sub-linear edge congestion to provide these services. 

	% THink about PKI as design and RSA and el-gamal are solutions to it.
	% As we saw in the previous chapter SHIA can detect the malicious activity in the network with only sub-linear edge congestion.
	% So, we built our work on top of SHIA which can detect an adversary.


	% For that we want non-repudiation

	% If there is any malicious activity detect it

	% Locate an adversary	
	% fig:Commitment payload of C to detect an adversary in the network with least amount of overhead in terms of BW and computation.
	% We wanted to achieve that with as realistic as possible. 


% \section{Design Goals}


% 	% In the previous chapter, we saw that SHIA limits the adversary's ability to manipulate the aggregation result with the tightest bound possible.
% 	% But, SHIA uses truncated sum as an aggregate function which is resilient according to the Wagner \cite{wagner2004resilient}.
% 	% It does not help accurately calculating the sum aggregate. 
% 	% Furthermore, SHIA does not help detecting and revoking the from the of Sensor Node malicious aggregate no network.
% 	% SHIA does not require prior knowledge of network topology and works on hierarchical sensor network which might include multiple malicious sensor nodes, with only suboptimal congestion overhead.
% 	% SHIA helps the sensor node verify, its reported sensor reading was aggregated correctly or not, by an an aggregate node.
% 	% If an an aggregate node has tampered with the reported sensor reading then the relevant sensor node can detect the tampering and raise an alarm.
% 	We develop the aggregation protocol which works for any random hierarchical sensor networks with a single or multiple adversaries.
% 	It works on the resilient as well as non-resilient aggregation functions Defined in \ref{def:resilient}, without compromising any desired security properties.
% 	It
% 	It prevents masquerade attacks in the network so this protocol can easily be applied to do the voting scheme in the network.

	% We want to achieve all the mentioned goals by inducing only $O(\delta \log^{k} n)$ node congestion in the aggregation tree; where $n$ is the number of nodes in the network, $\delta$ is the fanout of any node in the aggregation tree, and $k$ is the variable.
	% Ideally, we want $k=0$, in the case of SHIA and our protocol $k=2$, which will be explained in the later chapter.
	% % which is built on top of commitment tree generation of SHIA mentioned in the previous chapter.

% \section{Security Mechanisms}
	To detect any malicious activity in the network which is an important part of achieving data-integrity in the network we use \textit{Hash Functions} as security mechanism.
	To protect against any active security attacks, provide authentication and non-repudiation security services we use \textit{Digital Signatures} as the security mechanisms.
	Both of these mechanisms are described in detail in Chapter \ref{cha:Cryptography tools}.	