\chapter{In-network Data Aggregation Overview} % (fold)
\label{cha:In-network Data-Aggregation Overview}

\section{In-network data aggregation}
	Sensor networks are being used in scientific data collection, fire alarm systems, traffic monitoring, wildfire tracking, wildlife monitoring and many other applications.
	In sensor networks, thousands of sensor nodes interact with the physical environment and collectively monitors an area, generating a large amount of data to be transmitted and reason about.
	The sensor nodes in the network often have limited resources, such as computation power, memory, storage, communication capacity and most significantly, battery power.
	Also, data communication between nodes consumes a large portion of the total energy consumption. 
	The in-network data aggregation reduces the energy consumption by eliminating redundant data being transmitted to the base station.
	For example, in-network data aggregation of the $\textit{SUM}$ function can be performed as follows. 
	Each intermediate sensor node in the network forwards a single sensor reading containing the sum of all the sensor readings from all of its descendants, rather than forwarding each descendants sensor reading one at a time to the base station.
	It is shown that the energy savings achieved by in-network data-aggregation are significant \cite{madden2002tag}.
	The in-network data aggregation approach requires the sensor nodes to do more computations.
	But studies show that data transmission requires more energy than data computation. 
	Hence, in-data aggregation is an efficient and widely used approach for saving bandwidth by doing less communications between sensor nodes and ultimately giving longer battery life to sensor nodes in the network.

	We define following terms to help us define the goals of in-network data-aggregation approach.
	\begin{definition}\label{def:payload}
		\textbf{Payload} is the part of the transmitted data which is the fundamental purpose of the transmission, to the exclusion of information sent with it such as metadata solely to facilitate the delivery.
	\end{definition}
	\begin{definition}\label{def:information-rate}
		\textbf{Information-rate} for a given node is the ratio of the \payloads, number of \payloads\ sent divided by the number of \payloads\ received.
	\end{definition}
	The goal of the aggregation process is to achieve lowest possible \informationRate.
	In the following sections, we show that reducing \informationRate\ makes the intermediate (aggregator) sensor nodes more powerful.
	Also, it makes aggregated \payload\ more fragile and vulnerable to various security attacks.
	We describe bandwidth analysis of different in-network aggregation approaches.

\section{Bandwidth analysis}
	Congestion is widely used parameter while doing bandwidth analysis of networking applications. 
	The congestion for any given node is defined as follows:
	\begin{equation}
		Congestion = edgeCongestion * fanout
	\end{equation}
	Congestion is very useful factor while analyzing sensor network as it measures how quickly the sensor nodes will exhaust their batteries \cite{madden2003design}. 
	To transmit a $k$ - bit packet at distance $d$, the energy dissipated is:
	\begin{equation}
		E_{tx}(k, d) = E_{elec} * k + \varepsilon_{amp} * k * d^{2}
	\end{equation}
	and to receive the k - bit packet, the radio expends
	\begin{equation}
		E_{rx}(k) = E_{elec} * k
	\end{equation}
	For $\mu Amp$\ wireless sensor, $E_{elec} = 50nJ/b$\ and $\varepsilon_{amp} = 100pJ/b/m^2$. \textcolor{red}{cite papers and add more details on the equations}

	Also, some nodes in the sensor network have more congestion than the other. 
	The highly congested nodes are the most important to the the network connectivity, for example, the nodes closer to the base station are essential for the network connectivity.
	The failure of the highly congested nodes may cause the network to partition even though other nodes in the network have high battery levels.
	A low congestion on the highly congested nodes is thus desirable even though it costs more congestion within the overall sensor network. 

	\textit{SHIA} tries to achieve uniform congestion in the network.	

\section{Security in In-network data aggregation}
	In-network data aggregation approach saves bandwidth by transmitting less \payloads\  between sensor nodes but it gives more power to the intermediate aggregator sensor nodes. 
	For example, a malicious intermediate sensor node who is doing aggregation over all of its descendants \payloads, needs to tamper with only one aggregated \payload\ instead of tampering with all the \payloads\ received from all of its descendants. 
	It means a malicious intermediate sensor node needs to do less work to skew the final aggregated \payload.
	Also, an adversary controlling few sensor nodes in the network can cause the network to return unpredictable \payloads, making an entire sensor network unreliable.
	Notice that, the more descendants an intermediate sensor node has the more powerful it becomes.
	Despite the fact that in-network aggregation makes an intermediate sensor nodes more powerful, some aggregation approaches requires strong network topology assumptions or honest behaviors from the sensor nodes.
	For example, in-network aggregation schemes in \cite{yao2002cougar, madden2003design} assumes that all the sensor nodes in the network are honest. Secure Information Aggregation (SIA) of \cite{przydatek2003sia}, provides security for the network topology with a single-aggregator model.  

	Secure hierarchical in-network aggregation (\textit{SHIA}) in sensor networks \cite{chan2006secure} presents the first and provably secure sensor network data aggregation protocol for general networks and multiple adversaries. 
	We discuss the details of the protocol in the next chapter. 
	\textit{SHIA} limits the adversary's ability to tamper with the aggregation result with the tightest bound possible but it does not help detecting an adversary in the network.
	Also, we claim that same upper bound can be achieved with compact label format defined in the next chapter.