\chapter{november}

Misc. topics to write about:

\textit{Why do you want to communicate an entire aggregation tree to the querier ?}
	If the querier knows the entire aggregation tree and also if it knows the protocol which all the sensor nodes will be running then the querier can simulate the commitment trees on its own. Because of that we do not have to communicate the commitment tree every time we run the protocol which saves a lot of communications in the network. Also, note the fact that aggregation tree does not change often so the communication required to send the aggregation tree is negligible over time.

\textit{How to communicate an entire aggregation tree to the querier ?}
	The base station in the aggregation tree needs to know the entire network topology.
	It will relay that information to the querier.

\textit{How does the base station know the entire aggregation tree topology ?}
	If every sensor nodes has a small table containing the path to reach to the certain destination then the base station can ask for this information to the individual sensor nodes. While it is receiving this information it can relay the same information to the querier. Note: the base station is also a simple sensor node like all other nodes it can not store all the forwarding tables so it will relay those table information directly to the querier and querier can make big table containing the information related to the aggregation tree.

\textit{Mobility}
	You can talk about the aggregation tree topology is mobile. It's increasingly mobile topology not leap mobility.

\textit{Caching of certificates}
		Certificates are sent only once for the first time. They are cached for subsequent communications. Every node in the tree needs to know the certificates of all the root nodes in its forest.

\textit{Why does the internal vertex in the commitment tree need to send what it received and what it sent to its parent ?}
	To detect a cheater, if an internal vertex send ( to the querier ) only the values which it sent to its parent in the commitment tree then it is no value to the querier. Because the querier can not verify that value and the signature. For the querier to verify the aggregated data and its signature it needs both the values over which aggregation has happend. 

\textit{Why don't you need backward signatures ?} Because according to the protocol, every parent checks its children's message and its signature. If those two do not match then it will not accept the message.

\textit{Do you need signature on forest ? If yes, then why ? If no, then why ?}

\textit{Analyses of being root in as many tree as possible:}
	
	\begin{itemize}
		\item \textit{Bandwidth perspective}
	
			\textit{Off path values}
				
				It takes same bandwidth (same hop counts) to distribute off path values in include itself or exclude itself stratergy. You can have inductive argument for it to prove it.

			\textit{Certificates}
				Parent node needs to deal with less nodes in the aggregation tree means it needs less certificates, means less memory storage. For example, in pseudo palm tree case if we use include it self streategy then it is possible that one node has to propagate its value from the bottom to the top of the tree. It means all the intermediate nodes need to know its certificate. This can be avoided by using exclude itself( being root in as many possible tree as possible ) stratergy.

		\item \textit{Security perspective}

				Exclude itself stratergy is more secure in the sense that aggregator needs to partner with two nodes to achieve cheating. If it includes itself then it has to partner with only one node which is relatively easy.

	\end{itemize}


\textit{Why do we need authenticated broadcast from the querier ?}

\textit{Significance of Nonce}

\textit{Why do we need public key infrastructure ?}

\textit{Why don't we use aggregation tree as commitment tree ?}

\textit{Why is commitment tree binary and not n-ary ? (proof)}

\begin{algorithm}[H]\label{number3} \caption {CommitmentTreeGeneration}
	\begin {algorithmic}[1]

		\STATE Starting with highest depth in decreasing order 

			\FORALL {\node \   $\in$ \aggregationTree }

					\STATE Create a \node $_{0}$, a \sign $_{\cal{N}}$ ( \node$_{0}$ ) and attach it to \forest $_{ \cal{N } }$

					\IF {\node \ has \children}

						\FORALL {\children \ $\in$ \node}

							\FORALL {\treeRoot \ $\in$ \forest $_{ \mathcal{CHILDREN}}$}

								\IF {\node \ has \cert$_{\mathcal{R}}$}

									\IF {\cert$_{\mathcal{R}}$ \ is verified by \node} 

										\STATE \node \ gets $\mathcal{R}$, \sign$_{\mathcal{R}}$
										\STATE \node \ verifies $\mathcal{R}$, \sign$_{\mathcal{R}}$

										\IF {\treeRoot \ is Verified by \node}

											\STATE Add \treeRoot \ to \forest$_{\mathcal{N}}$
											\STATE Run Huffman coding on \forest$_{\mathcal{N}}$
									
										\ENDIF

									\ENDIF

								\ENDIF

							\ENDFOR

						\ENDFOR

					\ENDIF

			\ENDFOR

	\end{algorithmic}

\end{algorithm}

\begin{algorithm}
\caption{Pseudo algorithm to detect a cheater}
	\begin{algorithmic}[1]
			\STATE The querier finds out the nodes who are complaining 
			\STATE The querier asks the complainers to send their reading \& \sign 
			\STATE The querier finds possible cheaters based on complaines
			\STATE The querier asks possible cheaters to send the messages \& \sign s they received and also the messages \& \sign s they send. It can also asks the parents of the possible cheaters to do so
			\STATE The querier determines the cheater 
	\end{algorithmic}
\end{algorithm}

\textit{Properties of commitment tree and aggregation tree}

	If you have $O(n)$ children then you need atleast $\Omega(n)$ \& at max $O(nlog(n))$ certificates.

	If you have $O(n)$ descendents then you need $\Omega(log(n))$  \& at max $O(nlog(n))$ certificates.

\forest